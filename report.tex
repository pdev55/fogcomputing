\documentclass{article}

\usepackage{listings}
\usepackage{color} %red, green, blue, yellow, cyan, magenta, black, white
\definecolor{mygreen}{RGB}{28,172,0} % color values Red, Green, Blue
\definecolor{mylilas}{RGB}{170,55,241}

\usepackage{graphicx}
\usepackage{subcaption}

\usepackage{float}

\usepackage{geometry}
\geometry{a4paper,margin=2.5cm}
\setlength{\parindent}{0pt} %no indentation between paragraphs
\setlength{\parskip}{4pt} %space between paragraphs

\title{CSSE7014 Distributed Computing \\
Assignment 2 \\
Semester 1, 2017}
\author{Paul Kogel (44644743), Ramdas Ramani (44743767), Andi Nuruljihad (44159069)}

\begin{document}

\maketitle

\pagebreak
\tableofcontents\thispagestyle{plain}

\pagebreak

\section{Introduction}

Fog computing is a new, exciting computing paradigm \cite{bonomi2012fog}.

The introduction is clear with several definitions of the computing paradigm under study for comparison. The structure of the report is presented.

\section{Architectures and Models}
Compare and contrast different architectures and models with examples to back the arguments.

RAM

\pagebreak

\section{Common Issues}
% Comments on issue related to communication paradigms, fault tolerance, consistency, reliability, etc.
% Marking criteria: Quality discussion and explanation on relevant issues as required with clear examples. Discussions of potential enhancement to address any performance issues are provided.

Though still an emerging field, previous research, such as Yi et al.'s ``survey on fog computing'' \cite{yi2015survey}, has been able to identify multiple potential issues related to fog computing. In this section, we summarise their main findings, and provide additional discussion and research related to them whenever possible.

To improve clarity, we organise issues around 5 main areas: networking, optimal use of resources, fault tolerance, application development, and security and privacy. Note that we focus purely on technical issues. Business-related aspects, such as implementation of a viable business model, and billing mechanisms, are not covered.

\subsection{Networking}
In order for the fog to function properly, the network has to provide nodes with connectivity, and additional network services, such as routing. The particular nature of the fog, though, makes the implementation of these functions difficult. 

For example, the network has to be highly scalable, providing support for a large number of potential nodes. In addition, it should account for constant topology changes due to node mobility. Ensuring that the system is fully distributed is also an important aspect \cite{yi2015survey}.

Using virtualisation mechanisms, such as SDN, has been deemed as viable solution to these issues \cite{yi2015survey}. In their proposal of a general architecture for the fog, Bonomi et al. \cite{bonomi2014fog} state that the fog should use virtualisation for ``key resources'', including networking. Providing an implementation of SDN for the fog, however, is still an open issue \cite{yi2015survey}. Partly, this appears to be the case because SDN itself does not put ``a high emphasis'' on distribution \cite{peng2016fog}.


\subsection{Optimal resource use}
\label{sub_opt_res_use}

As stated before, the fog is highly heterogeneous. This heterogeneity is greatly reflected by different degrees of resource availability throughout the system. Important resources are storage, computation power and bandwidth. For example, in parts of the system, available bandwidth might be high due to the presence of more powerful network links, while in others, it can be a scare resource.

Naturally, these resources should be ``optimally'' used. However, the actual optimisation goal is highly dependent on the use case: for example, in a real-time application, the main objective is to ensure a small delay. Using more bandwidth or computation power to meet this goal is a valid trade-off. For a computation-heavy application running on a mobile device, in contrast, reducing the amount of computation performed locally on the device might be most important.

In their paper, Yi et al. \cite{yi2015survey} present several strategies that might be used to optimise resource use under different circumstances. 
%
Firstly, they suggest that the adequate placement of data can help optimising bandwidth use. In the previously given example of a real-time application, storing data on nodes that are well-connected to the consumer could significantly reduce delay. This placement, however, has to take the dynamic nature of the fog into account. If a node changes location, for example, data placement on the same node can result in small latencies at one time, but introduce great delays at another time. Even if the location of the consumer does not change, available bandwidth at a link might, e.g. if more nodes are interested in the data.
%
Besides placing data, the authors also suggest to place computation. Using ``computation offloading'', an operation can be partly or fully delegated to a different node in the network. In the aforementioned example of the computation-heavy application, for instance, a more powerful node could perform most of the computation-intense work. Determining which parts of the computation to offload to which nodes, though, can be challenging. As for data placement, this is largely due to the dynamic nature of the fog.
%
Lastly, they present different methods based on the concept of adjusting the network topology. For example, they suggest that effectively choosing the relay nodes for one or multiple endpoints could help reducing delay, while increasing throughput. Again, however, constant changes in the environment, especially the topology, make an implementation challenging.

Focusing less on choosing locations for data and operations, and more on resources actually available at a given node, Aazam and Huh \cite{aazam2015dynamic} present a resource management method that has been developed especially for the dynamic environment of the fog. At its core, their method predicts the resources required by a consumer for the use of a specific service, and uses these predictions to give ``guarantees'' about resource availability. For example, a consumer might get a guarantee of 80\% availability for a particular service, meaning that it will have access to all resources required to run the service for most of the time. Predictions are largely dependent on past consumer behaviour. If the node in the previous example had, for instance, frequently disconnected from the service provider, its resource guarantee would be lower. Basically, this means that if not sufficient resources are available, they would preferably be given to nodes that make better use of them. As it can be easily seen, this makes resource allocation considerably fair.

\subsection{Fault tolerance}
As describe before, the fog mainly uses unreliable wireless network links. In addition, nodes are highly mobile. Being able to ensure availability of services, and provide reliability in general are therefore important aspects.

To improve service availability, Yi et al. \cite{yi2015survey} suggest to adjust the network topology (see section \ref{sub_opt_res_use}). For instance, they present the idea of dividing a network into several clusters, with each cluster centred around a ``rich-resource'' node.

Traditionally, reliability in a distributed system can be provided by the means of techniques such as checkpointing or rescheduling (see \cite{tanebaum2013}). According to Yi et al. \cite{yi2015survey}, though, most of these techniques are unfit for the fog, as they introduce too much delay. They conclude that replication might work, but they expect it to be difficult to implement due to the distributed nature of the system. Additional research on the topic does not seem to exist. Madsen et al. \cite{madsen2013reliability} claim to provide such, but fail to give any actual fog computing-related insights.

%"However, taking a checkpoint is often a costly operation and may have a severe performance penalty." (p. 364)


\subsection{Application Development}
As stated in section XX, the fog is dynamic in regards to network topology, and resource availability. In addition, fog nodes might run on different platforms and system architectures \cite{yi2015survey}. Developing applications that are able to run in this environment, and provide high compatibility, can be expected to be difficult. 

To ease development, Bonomi et al. \cite{bonomi2014fog} propose a ``fog abstraction layer'' that hides the underlying heterogeneity, and provides developers with a ``uniform and programmable interface''. Yi et al. make a similar suggestion by calling for a ``unified interfacing and programming model'' \cite{yi2015survey}.

Due to issues mentioned in the beginning, though, we expect that the implementation of such a layer is challenging.

%TODO: what about cisco?

\subsection{Security and Privacy}
Many applications that have been proposed for fog computing are safety-critical, and/or process sensitive data. For example, in vehicle-to-vehicle communication, an insecure system that allows attackers to remotely control the car could have disastrous consequences. In home automation, users might be worried about giving third parties insights into their daily routine.

%Authentication/access control
Stojmenovic and Wen \cite{stojmenovic2014fog} find that providing authentication throughout the system is one of the ``main security issues'' for fog computing. As an example, they describe a smart meter that is modified by a user, and reports then, due to a lack of proper authentication, false readings. As a possible solution, they suggest encryption at node-level. For this, the meter would encrypt its data, and another node would decrypt it before further forwarding the data. Similarly to this, the OpenFog consortium deems access control (to which it counts authentication) as ``key to building a secure system'' \cite{openfogconsortium2017}. 

%Root of trust
In addition to advocating access control, the consortium's reference architecture for fog computing also defines a hardware component called ``root of trust'' that is ``at the heart of the [...] security of the fog node''. This component is tamper proof, and required to be implemented by every fog node. It provides security by creating a ``chain of trust'', i.e., selecting other components such as hardware, software, or other nodes that it considers trustworthy. If a component is compromised, like the smart meter in the example above, it would not gain trust from the root, and therefore not do any harm.
%stresses the importance of providing ``end-to-end security'' in general. To establish that, it

Though access control and the chain of trust promise to provide a solid foundation to a secure fog system, they are both rather general methods. In order to improve security in a given context, it has been suggested in \cite{openfogconsortium2017} to select additional measures based on the specific use case.

To protect privacy, Yi et al. \cite{yi2015survey} suggest to run ``privacy-preserving'' algorithms before data is transferred from the fog to the cloud. As examples, they mention techniques based on differential privacy and homomorphic encryption. Gerla \cite{gerla2012vehicular} makes an interesting point by stating that moving processing from the cloud to mobile devices alone gives users more control over their data. It can be easily seen, however, that this requires the implementation of adequate control mechanisms. The aforementioned reference architecture \cite{openfogconsortium2017} vaguely describes ``privacy attributes'' that a user can assign to his/her data, suggesting that these might be used to control its use.

\pagebreak

\section{Applications}
%Various examples (across different disciplines) provided with clear arguments why they are relevant.

Fog computing was conceptualized as an extension of the cloud to address services and applications for which the cloud paradigm is not entirely suitable \cite{bessis2014big}. As a relatively new model, the potential applications and likely infrastructure and design challenges for the fog are still being explored. However, there is a wide range of possible uses of a paradigm that enables real-time, low-latency processing, reduces bandwidth costs, with the benefits of improved security and governance.

\subsection{Real-time Health Monitoring}
Wireless Body Area Networks (WBAN) is an important technology in healthcare Internet-of-Things (IoT) applications that allows for the unobtrusive monitoring and recording of various vital signs of a patient in real-time. Many health-monitoring systems employ cloud servers for their low cost, processing power, and high storage volume. The primary drawbacks of relying on the cloud paradigm for such a system are the high latency and the volume of data transmitted over the network by such a large number of sensor nodes. For such a system that is highly dependent on accurate, real-time processing of large amounts of data, there is a need for the reduction of transmitted data that still guarantees quality of service (QoS) \cite{gia2015fog}.

Gia et al. \cite{gia2015fog} suggest the ``provision of an extra layer in between a conventional gateway and a remote cloud server''. This added layer, the ``fog layer'', would serve to preprocess data from the various sensor nodes, expediting the response time of vital applications while simultaneously reducing the volume of data transmitted to the cloud.

\subsection{Security and Surveillance}
Surveillance and security camers generate massive volumes of data; a single may create over one terabyte of high-definition video data per day \cite{openfogconsortium2017visualsecurity}. Security decisions must be made rapidly as situations may arise where a delay in response can result in a risk to public safety. The latency and reliability issues inherent in cloud solutions, where such large amounts of security information must be transmitted to a remote location, makes the cloud paradigm an imperfect solution for such a problem.

Fog-based deployments provide real-time tracking and anomaly detection from a location much closer to the edge. Fog nodes can be designed to intelligently partition video processing between the various edge devices and the cloud. Video analytics can be performed at fog nodes, reducing latency and allowing for quicker response times in emergency situations. Relevant data can then be sent to the cloud to allow for historical analysis over long periods of time and enable data sharing between multiple locations (for example, the sharing of data between airports or multiple locations of a hotel chain).

Fog computing provides a solution that is scalable and more secure. Recent camera data is stored at fog gateways located close to the edge and long-term data is stored in the cloud to allow for analysis of long-term trends. The addition of an extra layer between the edge and the cloud suggests additional security steps in the process of sending data from edge devices to the cloud. Raw camera data is analyzed at the fog gateways, where rules can be established to determine which data must be sent to the cloud and which can be stored locally, and video processing can be partitioned intelligently between the cloud, the fog gateway, and the edge devices.

\subsection{Smart Cities - Traffic Congestion Management}
Traffic management is an ever-growing challenge faced by major cities where road congestion can be attributed to lost productivity, slower response times from emergency services, and high carbon emissions. Various factors that contribute to traffic congestion simply can not be predicted, such as accidents and weather. In addition, traffic management is a challenge that spans multiple jurisdictions; the development and implementation of potential solutions takes place in isolation within each department, hindering information sharing and integration initiatives \cite{openfogconsortium2017trafficmanagement}.

Proposed cloud solutions in Smart Cities, while suggesting a single cloud that connects these many departments, in reality involve multiple clouds for traffic management. As different departments will be responsible for the implementation of their own solutions to the traffic congestion problem for areas under their jurisdiction, they will potentially work with any number of clouds. This segmented approach, where each piece of the solution is developed and deployed in isolation, can result in the accumulation of redundant data and is an obstacle to data sharing between departments which in turn creates delays in response.

Fog computing allows for any-to-any communication paths without disrupting existing edge-to-cloud communications \cite{outofthefog2017}. CCTV, camera, electronic signage, and traffic light data captured by roadside or local fog nodes can be analyzed and used locally and then sent to regional fog nodes to be related to other data. Fog gateways can be added to share information across individual networks, allowing for all relevant departments to share a single, synchronized picture of road-side conditions as they change. In the future, as manufacturers install fog nodes in their vehicles, each vehicle will serve as an edge device that connects with other devices on the road. All of this data in aggregate can be used by municipalities to provide a far more detailed view of traffic conditions, allowing for greater control of road conditions and much quicker response times.

\pagebreak

\renewcommand{\refname}{\section{References}}
\bibliographystyle{ieeetr}
\bibliography{lib}

\end{document}
